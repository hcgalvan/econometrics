% Options for packages loaded elsewhere
\PassOptionsToPackage{unicode}{hyperref}
\PassOptionsToPackage{hyphens}{url}
%
\documentclass[
]{article}
\usepackage{amsmath,amssymb}
\usepackage{iftex}
\ifPDFTeX
  \usepackage[T1]{fontenc}
  \usepackage[utf8]{inputenc}
  \usepackage{textcomp} % provide euro and other symbols
\else % if luatex or xetex
  \usepackage{unicode-math} % this also loads fontspec
  \defaultfontfeatures{Scale=MatchLowercase}
  \defaultfontfeatures[\rmfamily]{Ligatures=TeX,Scale=1}
\fi
\usepackage{lmodern}
\ifPDFTeX\else
  % xetex/luatex font selection
\fi
% Use upquote if available, for straight quotes in verbatim environments
\IfFileExists{upquote.sty}{\usepackage{upquote}}{}
\IfFileExists{microtype.sty}{% use microtype if available
  \usepackage[]{microtype}
  \UseMicrotypeSet[protrusion]{basicmath} % disable protrusion for tt fonts
}{}
\makeatletter
\@ifundefined{KOMAClassName}{% if non-KOMA class
  \IfFileExists{parskip.sty}{%
    \usepackage{parskip}
  }{% else
    \setlength{\parindent}{0pt}
    \setlength{\parskip}{6pt plus 2pt minus 1pt}}
}{% if KOMA class
  \KOMAoptions{parskip=half}}
\makeatother
\usepackage{xcolor}
\usepackage[margin=1in]{geometry}
\usepackage{color}
\usepackage{fancyvrb}
\newcommand{\VerbBar}{|}
\newcommand{\VERB}{\Verb[commandchars=\\\{\}]}
\DefineVerbatimEnvironment{Highlighting}{Verbatim}{commandchars=\\\{\}}
% Add ',fontsize=\small' for more characters per line
\usepackage{framed}
\definecolor{shadecolor}{RGB}{248,248,248}
\newenvironment{Shaded}{\begin{snugshade}}{\end{snugshade}}
\newcommand{\AlertTok}[1]{\textcolor[rgb]{0.94,0.16,0.16}{#1}}
\newcommand{\AnnotationTok}[1]{\textcolor[rgb]{0.56,0.35,0.01}{\textbf{\textit{#1}}}}
\newcommand{\AttributeTok}[1]{\textcolor[rgb]{0.13,0.29,0.53}{#1}}
\newcommand{\BaseNTok}[1]{\textcolor[rgb]{0.00,0.00,0.81}{#1}}
\newcommand{\BuiltInTok}[1]{#1}
\newcommand{\CharTok}[1]{\textcolor[rgb]{0.31,0.60,0.02}{#1}}
\newcommand{\CommentTok}[1]{\textcolor[rgb]{0.56,0.35,0.01}{\textit{#1}}}
\newcommand{\CommentVarTok}[1]{\textcolor[rgb]{0.56,0.35,0.01}{\textbf{\textit{#1}}}}
\newcommand{\ConstantTok}[1]{\textcolor[rgb]{0.56,0.35,0.01}{#1}}
\newcommand{\ControlFlowTok}[1]{\textcolor[rgb]{0.13,0.29,0.53}{\textbf{#1}}}
\newcommand{\DataTypeTok}[1]{\textcolor[rgb]{0.13,0.29,0.53}{#1}}
\newcommand{\DecValTok}[1]{\textcolor[rgb]{0.00,0.00,0.81}{#1}}
\newcommand{\DocumentationTok}[1]{\textcolor[rgb]{0.56,0.35,0.01}{\textbf{\textit{#1}}}}
\newcommand{\ErrorTok}[1]{\textcolor[rgb]{0.64,0.00,0.00}{\textbf{#1}}}
\newcommand{\ExtensionTok}[1]{#1}
\newcommand{\FloatTok}[1]{\textcolor[rgb]{0.00,0.00,0.81}{#1}}
\newcommand{\FunctionTok}[1]{\textcolor[rgb]{0.13,0.29,0.53}{\textbf{#1}}}
\newcommand{\ImportTok}[1]{#1}
\newcommand{\InformationTok}[1]{\textcolor[rgb]{0.56,0.35,0.01}{\textbf{\textit{#1}}}}
\newcommand{\KeywordTok}[1]{\textcolor[rgb]{0.13,0.29,0.53}{\textbf{#1}}}
\newcommand{\NormalTok}[1]{#1}
\newcommand{\OperatorTok}[1]{\textcolor[rgb]{0.81,0.36,0.00}{\textbf{#1}}}
\newcommand{\OtherTok}[1]{\textcolor[rgb]{0.56,0.35,0.01}{#1}}
\newcommand{\PreprocessorTok}[1]{\textcolor[rgb]{0.56,0.35,0.01}{\textit{#1}}}
\newcommand{\RegionMarkerTok}[1]{#1}
\newcommand{\SpecialCharTok}[1]{\textcolor[rgb]{0.81,0.36,0.00}{\textbf{#1}}}
\newcommand{\SpecialStringTok}[1]{\textcolor[rgb]{0.31,0.60,0.02}{#1}}
\newcommand{\StringTok}[1]{\textcolor[rgb]{0.31,0.60,0.02}{#1}}
\newcommand{\VariableTok}[1]{\textcolor[rgb]{0.00,0.00,0.00}{#1}}
\newcommand{\VerbatimStringTok}[1]{\textcolor[rgb]{0.31,0.60,0.02}{#1}}
\newcommand{\WarningTok}[1]{\textcolor[rgb]{0.56,0.35,0.01}{\textbf{\textit{#1}}}}
\usepackage{graphicx}
\makeatletter
\def\maxwidth{\ifdim\Gin@nat@width>\linewidth\linewidth\else\Gin@nat@width\fi}
\def\maxheight{\ifdim\Gin@nat@height>\textheight\textheight\else\Gin@nat@height\fi}
\makeatother
% Scale images if necessary, so that they will not overflow the page
% margins by default, and it is still possible to overwrite the defaults
% using explicit options in \includegraphics[width, height, ...]{}
\setkeys{Gin}{width=\maxwidth,height=\maxheight,keepaspectratio}
% Set default figure placement to htbp
\makeatletter
\def\fps@figure{htbp}
\makeatother
\setlength{\emergencystretch}{3em} % prevent overfull lines
\providecommand{\tightlist}{%
  \setlength{\itemsep}{0pt}\setlength{\parskip}{0pt}}
\setcounter{secnumdepth}{-\maxdimen} % remove section numbering
\ifLuaTeX
  \usepackage{selnolig}  % disable illegal ligatures
\fi
\usepackage{bookmark}
\IfFileExists{xurl.sty}{\usepackage{xurl}}{} % add URL line breaks if available
\urlstyle{same}
\hypersetup{
  pdftitle={Trabajo Practico 2 - Econometria},
  hidelinks,
  pdfcreator={LaTeX via pandoc}}

\title{Trabajo Practico 2 - Econometria}
\author{}
\date{\vspace{-2.5em}}

\begin{document}
\maketitle

\subsubsection{a) Generación de
Variables}\label{a-generaciuxf3n-de-variables}

Generar 4 variables: una que represente el salario semanal en pesos, dos
variables binarias: género y una que represente si estudió en una
universidad pública o privada.

Según un estudio de la consultora Ekonomía en 2021, los graduados de
universidades privadas en Argentina ganaban en promedio un 27\% más que
los graduados de universidades públicas al inicio de sus carreras
laborales.

Esta brecha salarial tendía a disminuir con los años de experiencia,
pero aún se mantenía una diferencia del 17\% a favor de los egresados de
instituciones privadas después de 10 años de haber obtenido el título.

Las carreras con mayores brechas salariales eran las relacionadas con
negocios, administración y finanzas, mientras que en áreas como
ingeniería y ciencias exactas, la diferencia era menor.

Cuál es la diferencia de sueldos en base a la experiencia en egresados
universidad en argentina

\begin{enumerate}
\def\labelenumi{\arabic{enumi}.}
\tightlist
\item
  Estudio de PwC Argentina (2021):
\end{enumerate}

\begin{itemize}
\tightlist
\item
  Los salarios iniciales de recién graduados universitarios eran un 35\%
  menores que los de profesionales con 5 años de experiencia.
\item
  Después de 10 años de experiencia, los salarios aumentaban en promedio
  un 80\% en comparación con los ingresos iniciales.
\end{itemize}

\begin{enumerate}
\def\labelenumi{\arabic{enumi}.}
\setcounter{enumi}{1}
\tightlist
\item
  Informe del Observatorio de Inserción Profesional de la UBA (2020):
\end{enumerate}

\begin{itemize}
\tightlist
\item
  Los graduados de carreras de grado de la Universidad de Buenos Aires
  duplicaban sus ingresos después de 5 años de experiencia laboral.
\item
  La brecha salarial entre recién egresados y profesionales con más de
  10 años de experiencia podía llegar al 120\% en algunas disciplinas.
\end{itemize}

\begin{enumerate}
\def\labelenumi{\arabic{enumi}.}
\setcounter{enumi}{2}
\tightlist
\item
  Encuesta de Salarios de Profesionales de KPMG (2019):
\end{enumerate}

\begin{itemize}
\tightlist
\item
  Los salarios promedio de profesionales argentinos con menos de 2 años
  de experiencia eran un 45\% menores que los de aquellos con más de 10
  años de trayectoria laboral.
\end{itemize}

¿Cuál es la diferencia o la brecha entre varones y mujeres
universitarias con respecto al sueldo.?

algunos hallazgos e investigaciones previas sobre este tema en
Argentina, teniendo en cuenta que la situación podría haber cambiado:

\begin{itemize}
\item
  Según un estudio del Ministerio de Economía de Argentina en 2021, las
  mujeres egresadas de universidades ganaban en promedio un 27\% menos
  que sus pares hombres al iniciar sus carreras laborales.
\item
  El informe ``Brechas de Género en la Educación Superior'' del Banco
  Mundial (2020) encontró que la brecha salarial entre hombres y mujeres
  profesionales en Argentina oscilaba entre el 15\% y el 25\%,
  dependiendo del campo de estudio.
\item
  Una encuesta de la consultora Mercer (2019) reveló que la brecha
  salarial de género entre egresados universitarios en Argentina era de
  aproximadamente 22\% a favor de los hombres, incluso después de
  controlar por factores como experiencia, nivel educativo y puesto de
  trabajo.
\item
  Según datos del Observatorio de Inserción Profesional de la UBA
  (2018), las graduadas mujeres de carreras de grado de esa universidad
  percibían salarios iniciales un 18\% menores que sus compañeros
  varones.
\end{itemize}

¿Cuál es la varianza de sueldos entre graduados universitarios de
economía en argentina?

Sin embargo, puedo compartir algunos hallazgos generales de estudios
previos sobre la dispersión salarial en esta profesión en Argentina:

\begin{enumerate}
\def\labelenumi{\arabic{enumi}.}
\item
  Encuesta de Salarios de UBA Económicas (2020): Este informe reveló que
  el salario promedio mensual de los graduados de la Facultad de
  Ciencias Económicas de la Universidad de Buenos Aires oscilaba entre
  \$45.000 y \$90.000 pesos argentinos, dependiendo de la antigüedad
  laboral.
\item
  Estudio de Salarios de KPMG (2019):\\
  Para el área de Finanzas/Economía, este reporte indicó que los
  salarios brutos mensuales en Argentina iban desde \$38.000 para
  posiciones junior hasta \$180.000 para cargos gerenciales.
\item
  Reporte de PwC (2018): Según este informe, el rango salarial de
  economistas en empresas privadas argentinas variaba desde \$30.000
  para analistas hasta \$120.000 para puestos de jefaturas.
\end{enumerate}

Es importante destacar que estos datos pueden haber variado en los
últimos años debido a factores económicos, políticas salariales,
inflación y otros aspectos propios del mercado laboral argentino.

\begin{Shaded}
\begin{Highlighting}[]
\CommentTok{\# Generar variables y datos de muestra}
\CommentTok{\# set.seed(123)}
\FunctionTok{set.seed}\NormalTok{(}\FunctionTok{sample}\NormalTok{(}\DecValTok{1}\SpecialCharTok{:}\DecValTok{10000}\NormalTok{, }\DecValTok{1}\NormalTok{))}
\NormalTok{salario\_semanal }\OtherTok{\textless{}{-}}\DecValTok{300000} 
\CommentTok{\# experiencia(0 = Falso, 1 = Verdadero)}
\NormalTok{experiencia }\OtherTok{\textless{}{-}} \FunctionTok{sample}\NormalTok{(}\DecValTok{0}\SpecialCharTok{:}\DecValTok{10}\NormalTok{, }\DecValTok{1000}\NormalTok{, }\AttributeTok{replace =} \ConstantTok{TRUE}\NormalTok{)}
\CommentTok{\# Género (0 = Femenino, 1 = Masculino)}
\NormalTok{genero }\OtherTok{\textless{}{-}} \FunctionTok{sample}\NormalTok{(}\FunctionTok{c}\NormalTok{(}\DecValTok{0}\NormalTok{, }\DecValTok{1}\NormalTok{), }\DecValTok{1000}\NormalTok{, }\AttributeTok{replace =} \ConstantTok{TRUE}\NormalTok{)}
\CommentTok{\# Universidad (0 = Pública, 1 =  Privada)}
\NormalTok{universidad }\OtherTok{\textless{}{-}} \FunctionTok{sample}\NormalTok{(}\FunctionTok{c}\NormalTok{(}\DecValTok{0}\NormalTok{, }\DecValTok{1}\NormalTok{), }\DecValTok{1000}\NormalTok{, }\AttributeTok{replace =} \ConstantTok{TRUE}\NormalTok{)}

\CommentTok{\#valores}
\NormalTok{μ\_gen }\OtherTok{\textless{}{-}}\NormalTok{ (}\FloatTok{0.18} \SpecialCharTok{*}\NormalTok{ salario\_semanal)}
\NormalTok{μ\_exp }\OtherTok{\textless{}{-}}\NormalTok{ (}\FloatTok{0.05} \SpecialCharTok{*}\NormalTok{ salario\_semanal) }\CommentTok{\# 35\% en 5 años entonces 35/5 = 5\% por año}
\NormalTok{μ\_univ }\OtherTok{\textless{}{-}}\NormalTok{ (}\FloatTok{0.17} \SpecialCharTok{*}\NormalTok{ salario\_semanal)}
\NormalTok{error\_std\_dev }\OtherTok{\textless{}{-}} \DecValTok{50000} \CommentTok{\#tomando la mitad del valor salarial lo divido en 3 partes}

\CommentTok{\# Salario semanal en pesos}
\NormalTok{salario }\OtherTok{\textless{}{-}}\NormalTok{ salario\_semanal }\SpecialCharTok{+}\NormalTok{ μ\_gen }\SpecialCharTok{*}\NormalTok{ genero }\SpecialCharTok{+}\NormalTok{ μ\_exp }\SpecialCharTok{*}\NormalTok{ experiencia }\SpecialCharTok{+}\NormalTok{ μ\_univ }\SpecialCharTok{*}\NormalTok{ universidad }\SpecialCharTok{+} \FunctionTok{rnorm}\NormalTok{(}\DecValTok{1000}\NormalTok{, }\DecValTok{0}\NormalTok{, error\_std\_dev)}

\NormalTok{dat }\OtherTok{\textless{}{-}} \FunctionTok{data.frame}\NormalTok{(salario, genero, experiencia, universidad)}
\FunctionTok{plot}\NormalTok{(salario)}
\end{Highlighting}
\end{Shaded}

\includegraphics{tp2V3_files/figure-latex/unnamed-chunk-1-1.pdf}

\subsubsection{\texorpdfstring{\textbf{b) Modelo de
Regresión}}{b) Modelo de Regresión}}\label{b-modelo-de-regresiuxf3n}

Generar una ecuación en la cual el salario sea explicado por el género,
experiencia y por la interacción entre género y asistencia a universidad
pública. Interprete la salida de regresión.

\begin{itemize}
\tightlist
\item
  Chequear residuos para ver si existe heterocedasticidad. ¿Qué observa?
\item
  Realizar algún test de heterocedasticidad.
\item
  ¿Son robustos los errores estándares? ¿Cómo corrige los errores para
  que sean consistentes? ¿Qué observa al corregir?
\end{itemize}

\begin{Shaded}
\begin{Highlighting}[]
\CommentTok{\# Ajustar el modelo de regresión lineal}
\CommentTok{\# Salario semanal en pesos}
\CommentTok{\#set.seed(123)}
\NormalTok{modelo }\OtherTok{\textless{}{-}} \FunctionTok{lm}\NormalTok{(salario }\SpecialCharTok{\textasciitilde{}}\NormalTok{ genero }\SpecialCharTok{+}\NormalTok{ experiencia }\SpecialCharTok{+}\NormalTok{ genero }\SpecialCharTok{*}\NormalTok{ universidad)}
\FunctionTok{summary}\NormalTok{(modelo)}
\end{Highlighting}
\end{Shaded}

\begin{verbatim}
## 
## Call:
## lm(formula = salario ~ genero + experiencia + genero * universidad)
## 
## Residuals:
##     Min      1Q  Median      3Q     Max 
## -143336  -35968    -802   36393  189972 
## 
## Coefficients:
##                    Estimate Std. Error t value Pr(>|t|)    
## (Intercept)        303900.2     4063.8  74.782   <2e-16 ***
## genero              54245.8     4689.2  11.568   <2e-16 ***
## experiencia         14142.7      504.8  28.016   <2e-16 ***
## universidad         53011.5     4447.9  11.918   <2e-16 ***
## genero:universidad  -3690.0     6496.2  -0.568     0.57    
## ---
## Signif. codes:  0 '***' 0.001 '**' 0.01 '*' 0.05 '.' 0.1 ' ' 1
## 
## Residual standard error: 51180 on 995 degrees of freedom
## Multiple R-squared:  0.5809, Adjusted R-squared:  0.5792 
## F-statistic: 344.8 on 4 and 995 DF,  p-value: < 2.2e-16
\end{verbatim}

Interpretación:

\begin{enumerate}
\def\labelenumi{\arabic{enumi}.}
\item
  \textbf{Término constante (Intercept)}: El salario base estimado para
  una persona de género femenino (genero = 0), sin experiencia
  (experiencia = 0) y que asistió a una universidad publica (universidad
  = 1) es de 304731.3 unidades monetarias.
\item
  \textbf{Efecto del género (genero)}: Si una persona es de género
  masculino (genero = 1), su salario aumenta en 52255.0 unidades
  monetarias en comparación con una persona de género femenino (genero =
  0), manteniendo todo lo demás constante.
\item
  \textbf{Efecto de la experiencia (experiencia)}: Por cada año
  adicional de experiencia, el salario aumenta en 14442.9 unidades
  monetarias, manteniendo todo lo demás constante.
\item
  \textbf{Efecto de la universidad}: Si asistió a universidad privada
  (universidad=1), el salario aumenta en 49207.2 unidades monetarias,
  manteniendo todo lo demás constante.
\item
  \textbf{Interacción entre género y asistencia a universidad pública
  (genero:universidad)}: Si una persona es de género masculino (genero =
  1) y asistió a una universidad pública (universidad = 0), su salario
  disminuye -1285.1 en comparación con una persona de género femenino
  que asistió a una universidad privada.
\item
  \textbf{R-cuadrado ajustado}: El modelo explica aproximadamente el
  57.37\% de la variabilidad observada en el salario.
\item
  \textbf{Significancia estadística}: Los coeficientes son
  estadísticamente significativos (p-value \textless{} 0.001), lo que
  indica que los efectos del género y la experiencia, son relevantes
  para explicar el salario, excepto la interacción entre género y
  asistencia a universidad pública.
\end{enumerate}

\begin{Shaded}
\begin{Highlighting}[]
\NormalTok{e2}\OtherTok{=}\NormalTok{(modelo}\SpecialCharTok{$}\NormalTok{residuals)}\SpecialCharTok{\^{}}\DecValTok{2}
\FunctionTok{plot}\NormalTok{(modelo}\SpecialCharTok{$}\NormalTok{fitted.values,e2 , }\AttributeTok{main =} \StringTok{"e2 en funcion de Y ajustado"}\NormalTok{,}
     \AttributeTok{xlab =} \StringTok{"Y ajustado"}\NormalTok{, }\AttributeTok{ylab =} \StringTok{"resid al 2"}\NormalTok{)}
\end{Highlighting}
\end{Shaded}

\includegraphics{tp2V3_files/figure-latex/unnamed-chunk-3-1.pdf}

\begin{Shaded}
\begin{Highlighting}[]
\CommentTok{\# Gráfico de residuos vs. valores ajustados}
\FunctionTok{plot}\NormalTok{(modelo, }\AttributeTok{which =} \DecValTok{1}\NormalTok{)}
\end{Highlighting}
\end{Shaded}

\includegraphics{tp2V3_files/figure-latex/unnamed-chunk-4-1.pdf}

Observación: En el gráfico de residuos vs.~valores ajustados, se puede
apreciar que la varianza de los residuos es constante a lo largo de los
valores ajustados, lo que sugiere la presencia de homocedasticidad.

\paragraph{Prueba de
heterocedasticidad}\label{prueba-de-heterocedasticidad}

\begin{Shaded}
\begin{Highlighting}[]
\CommentTok{\# Prueba de Goldfeld{-}Quandt}
\FunctionTok{library}\NormalTok{(lmtest)}
\end{Highlighting}
\end{Shaded}

\begin{verbatim}
## Loading required package: zoo
\end{verbatim}

\begin{verbatim}
## 
## Attaching package: 'zoo'
\end{verbatim}

\begin{verbatim}
## The following objects are masked from 'package:base':
## 
##     as.Date, as.Date.numeric
\end{verbatim}

\begin{Shaded}
\begin{Highlighting}[]
\FunctionTok{gqtest}\NormalTok{(modelo)}
\end{Highlighting}
\end{Shaded}

\begin{verbatim}
## 
##  Goldfeld-Quandt test
## 
## data:  modelo
## GQ = 0.96886, df1 = 495, df2 = 495, p-value = 0.6375
## alternative hypothesis: variance increases from segment 1 to 2
\end{verbatim}

\textbf{La prueba de Goldfeld-Quandt} tiene un p-valor de 0.03754, lo
cual es menor que el nivel de significancia convencional de 0.05, lo que
sugiere evidencia de heterocedasticidad en los residuos.

\paragraph{Errores Robustos}\label{errores-robustos}

\begin{Shaded}
\begin{Highlighting}[]
\CommentTok{\# Errores estándar robustos a heterocedasticidad}
\FunctionTok{library}\NormalTok{(sandwich)}
\FunctionTok{coeftest}\NormalTok{(modelo, }\AttributeTok{vcov =} \FunctionTok{vcovHC}\NormalTok{(modelo, }\AttributeTok{type =} \StringTok{"HC1"}\NormalTok{))}
\end{Highlighting}
\end{Shaded}

\begin{verbatim}
## 
## t test of coefficients:
## 
##                     Estimate Std. Error t value Pr(>|t|)    
## (Intercept)        303900.23    4037.96  75.261   <2e-16 ***
## genero              54245.81    4729.36  11.470   <2e-16 ***
## experiencia         14142.72     495.17  28.561   <2e-16 ***
## universidad         53011.48    4273.61  12.404   <2e-16 ***
## genero:universidad  -3690.05    6543.17  -0.564   0.5729    
## ---
## Signif. codes:  0 '***' 0.001 '**' 0.01 '*' 0.05 '.' 0.1 ' ' 1
\end{verbatim}

\textbf{Observación}:

Al corregir con esta herramienta, los errores estándar de los
coeficientes aumentan en comparación con los errores estándar no
corregidos. Sin embargo, los coeficientes siguen siendo estadísticamente
significativos (p-valor \textless{} 0.01 para todos los coeficientes),
excepto la iteracción entre género y universidad.

Podemos concluir que los residuos del modelo de regresión lineal
muestran evidencia de heterocedasticidad, lo que viola uno de los
supuestos del modelo de regresión lineal clásico. Aunque los
coeficientes siguen siendo estadísticamente significativos después de
corregir los errores estándar, es importante tener en cuenta la
heterocedasticidad y, si es necesario, considerar modelos más apropiados
o realizar transformaciones en los datos para abordar este problema.

\subsubsection{c) Simulación de
Regresiones}\label{c-simulaciuxf3n-de-regresiones}

Simulación de Monte Carlo con 10 regresiones

\begin{Shaded}
\begin{Highlighting}[]
\CommentTok{\# Función para realizar una regresión y guardar la recta de regresión}
\NormalTok{realizar\_regresion\_y\_guardar\_recta }\OtherTok{\textless{}{-}} \ControlFlowTok{function}\NormalTok{(...) \{}
  \CommentTok{\# Generar datos de muestra}
  \FunctionTok{set.seed}\NormalTok{(}\FunctionTok{sample}\NormalTok{(}\DecValTok{1}\SpecialCharTok{:}\DecValTok{10000}\NormalTok{, }\DecValTok{1}\NormalTok{))  }\CommentTok{\# Semilla aleatoria}
\NormalTok{  salario\_semanal }\OtherTok{\textless{}{-}}\DecValTok{300000} 
  \CommentTok{\# Género (0 = Femenino, 1 = Masculino)}
\NormalTok{  genero }\OtherTok{\textless{}{-}} \FunctionTok{sample}\NormalTok{(}\FunctionTok{c}\NormalTok{(}\DecValTok{0}\NormalTok{, }\DecValTok{1}\NormalTok{), }\DecValTok{1000}\NormalTok{, }\AttributeTok{replace =} \ConstantTok{TRUE}\NormalTok{)}
  \CommentTok{\# Universidad (0 = Pública, 1 =  Privada)}
\NormalTok{  universidad }\OtherTok{\textless{}{-}} \FunctionTok{sample}\NormalTok{(}\FunctionTok{c}\NormalTok{(}\DecValTok{0}\NormalTok{, }\DecValTok{1}\NormalTok{), }\DecValTok{1000}\NormalTok{, }\AttributeTok{replace =} \ConstantTok{TRUE}\NormalTok{)}
  \CommentTok{\#valores}
\NormalTok{  μ\_gen }\OtherTok{\textless{}{-}}\NormalTok{ (}\FloatTok{0.18} \SpecialCharTok{*}\NormalTok{ salario\_semanal)}
\NormalTok{  μ\_univ }\OtherTok{\textless{}{-}}\NormalTok{ (}\FloatTok{0.17} \SpecialCharTok{*}\NormalTok{ salario\_semanal)}
\NormalTok{  error\_std\_dev }\OtherTok{\textless{}{-}} \DecValTok{50000} \CommentTok{\#tomando la mitad del valor salarial lo divido en 3 partes}

\NormalTok{  salario }\OtherTok{\textless{}{-}}\NormalTok{ salario\_semanal }\SpecialCharTok{+}\NormalTok{ μ\_gen }\SpecialCharTok{*}\NormalTok{ genero }\SpecialCharTok{+}\NormalTok{ μ\_univ }\SpecialCharTok{*}\NormalTok{ universidad }\SpecialCharTok{+} \FunctionTok{rnorm}\NormalTok{(}\DecValTok{1000}\NormalTok{, }\DecValTok{0}\NormalTok{, error\_std\_dev)}
    
  \CommentTok{\# Ajustar el modelo de regresión lineal}
\NormalTok{  modelo }\OtherTok{\textless{}{-}} \FunctionTok{lm}\NormalTok{(salario }\SpecialCharTok{\textasciitilde{}}\NormalTok{ genero }\SpecialCharTok{+}\NormalTok{ genero }\SpecialCharTok{*}\NormalTok{ universidad)}

    \CommentTok{\# Guardar la recta de regresión}
\NormalTok{  recta }\OtherTok{\textless{}{-}} \FunctionTok{data.frame}\NormalTok{(}
    \AttributeTok{genero =} \DecValTok{0}\SpecialCharTok{:}\DecValTok{1}\NormalTok{,}
    \AttributeTok{salario =} \FunctionTok{coef}\NormalTok{(modelo)[}\DecValTok{1}\NormalTok{] }\SpecialCharTok{+} \FunctionTok{coef}\NormalTok{(modelo)[}\DecValTok{2}\NormalTok{] }\SpecialCharTok{*}\NormalTok{ (}\DecValTok{0}\SpecialCharTok{:}\DecValTok{1}\NormalTok{) }\CommentTok{\# Ecuación para genero = 0, universidad = 1}
\NormalTok{  )}
  \FunctionTok{return}\NormalTok{(recta)}
\NormalTok{\}}

\CommentTok{\# Realizar la simulación de Monte Carlo}
\NormalTok{resultados }\OtherTok{\textless{}{-}} \FunctionTok{lapply}\NormalTok{(}\DecValTok{1}\SpecialCharTok{:}\DecValTok{10}\NormalTok{, realizar\_regresion\_y\_guardar\_recta)}

\CommentTok{\# Graficar todas las rectas de regresión en un solo gráfico}
\FunctionTok{plot}\NormalTok{(}\ConstantTok{NULL}\NormalTok{, }\AttributeTok{xlim =} \FunctionTok{c}\NormalTok{(}\DecValTok{0}\NormalTok{, }\DecValTok{1}\NormalTok{), }\AttributeTok{ylim =} \FunctionTok{c}\NormalTok{(}\DecValTok{290000}\NormalTok{, }\DecValTok{360000}\NormalTok{), }\AttributeTok{xlab =} \StringTok{"Genero"}\NormalTok{, }\AttributeTok{ylab =} \StringTok{"Salario"}\NormalTok{, }\AttributeTok{main =} \StringTok{"Rectas de Regresión"}\NormalTok{)}
\ControlFlowTok{for}\NormalTok{ (i }\ControlFlowTok{in} \DecValTok{1}\SpecialCharTok{:}\DecValTok{10}\NormalTok{) \{}
  \FunctionTok{lines}\NormalTok{(resultados[[i]]}\SpecialCharTok{$}\NormalTok{genero, resultados[[i]]}\SpecialCharTok{$}\NormalTok{salario, }\AttributeTok{col =}\NormalTok{ i)}
\NormalTok{\}}
\end{Highlighting}
\end{Shaded}

\includegraphics{tp2V3_files/figure-latex/unnamed-chunk-7-1.pdf}

\begin{Shaded}
\begin{Highlighting}[]
\CommentTok{\#legend("topright", legend = paste("Regresión", 1:10), col = 1:10, lty = 1)}
\end{Highlighting}
\end{Shaded}

En esta simulación, se define una función
\texttt{realizar\_regresion\_y\_guardar\_recta()} que realiza lo
siguiente:

\begin{enumerate}
\def\labelenumi{\arabic{enumi}.}
\item
  Genera datos de muestra aleatorios para las variables \texttt{genero},
  \texttt{universidad} y \texttt{salario}.
\item
  Ajusta un modelo de regresión lineal utilizando estas variables.
\item
  Calcula y guarda la recta de regresión correspondiente para el caso de
  \texttt{genero\ =\ 0} (femenino) y \texttt{universidad\ =\ 1}
  (privada).
\item
  Luego, se realiza la simulación de Monte Carlo ejecutando la función
  \texttt{realizar\_regresion\_y\_guardar\_recta()} 10 veces, y se
  guardan las rectas de regresión resultantes en la lista
  \texttt{resultados}.
\end{enumerate}

Finalmente, en un solo gráfico todas las rectas de regresión.

\begin{itemize}
\item
  El eje x representa el género 0=mujer, 1=varón.
\item
  El eje y representa el salario (de 290 a 350 mil unidades monetarias).
\item
  Cada recta de regresión se grafica con un color diferente (del 1 al
  10).
\end{itemize}

\begin{verbatim}
La salida de este código será un gráfico que muestra las 10 rectas de regresión diferentes en un solo gráfico, lo que permite visualizar la variabilidad en las rectas debido a la aleatoriedad en los datos generados en cada iteración de la simulación de Monte Carlo.
\end{verbatim}

\begin{Shaded}
\begin{Highlighting}[]
\CommentTok{\# Función para realizar una regresión y guardar la recta de regresión}
\NormalTok{realizar\_regresion\_y\_guardar\_recta\_2 }\OtherTok{\textless{}{-}} \ControlFlowTok{function}\NormalTok{(...) \{}
  \CommentTok{\# Generar datos de muestra}
  \FunctionTok{set.seed}\NormalTok{(}\FunctionTok{sample}\NormalTok{(}\DecValTok{1}\SpecialCharTok{:}\DecValTok{10000}\NormalTok{, }\DecValTok{1}\NormalTok{))  }\CommentTok{\# Semilla aleatoria}
\NormalTok{  salario\_semanal }\OtherTok{\textless{}{-}}\DecValTok{300000} 
  \CommentTok{\# Género (0 = Femenino, 1 = Masculino)}
\NormalTok{  genero }\OtherTok{\textless{}{-}} \FunctionTok{sample}\NormalTok{(}\FunctionTok{c}\NormalTok{(}\DecValTok{0}\NormalTok{, }\DecValTok{1}\NormalTok{), }\DecValTok{1000}\NormalTok{, }\AttributeTok{replace =} \ConstantTok{TRUE}\NormalTok{)}
  \CommentTok{\# Universidad (0 = Pública, 1 =  Privada)}
\NormalTok{  universidad }\OtherTok{\textless{}{-}} \FunctionTok{sample}\NormalTok{(}\FunctionTok{c}\NormalTok{(}\DecValTok{0}\NormalTok{, }\DecValTok{1}\NormalTok{), }\DecValTok{1000}\NormalTok{, }\AttributeTok{replace =} \ConstantTok{TRUE}\NormalTok{)}
  \CommentTok{\#valores}
\NormalTok{  μ\_gen }\OtherTok{\textless{}{-}}\NormalTok{ (}\FloatTok{0.18} \SpecialCharTok{*}\NormalTok{ salario\_semanal)}
\NormalTok{  μ\_univ }\OtherTok{\textless{}{-}}\NormalTok{ (}\FloatTok{0.17} \SpecialCharTok{*}\NormalTok{ salario\_semanal)}
\NormalTok{  error\_std\_dev }\OtherTok{\textless{}{-}} \DecValTok{50000} \CommentTok{\#tomando la mitad del valor salarial lo divido en 3 partes}

\NormalTok{  salario }\OtherTok{\textless{}{-}}\NormalTok{ salario\_semanal }\SpecialCharTok{+}\NormalTok{ μ\_gen }\SpecialCharTok{*}\NormalTok{ genero }\SpecialCharTok{+}\NormalTok{ μ\_univ }\SpecialCharTok{*}\NormalTok{ universidad }\SpecialCharTok{+} \FunctionTok{rnorm}\NormalTok{(}\DecValTok{1000}\NormalTok{, }\DecValTok{0}\NormalTok{, error\_std\_dev)}
    
  \CommentTok{\# Ajustar el modelo de regresión lineal}
\NormalTok{  modelo }\OtherTok{\textless{}{-}} \FunctionTok{lm}\NormalTok{(salario }\SpecialCharTok{\textasciitilde{}}\NormalTok{ genero }\SpecialCharTok{+}\NormalTok{ genero }\SpecialCharTok{*}\NormalTok{ universidad)}

 \CommentTok{\# Guardar la recta de regresión}
\NormalTok{  recta }\OtherTok{\textless{}{-}} \FunctionTok{data.frame}\NormalTok{(}
    \AttributeTok{interaccion =} \FunctionTok{c}\NormalTok{(}\DecValTok{0}\NormalTok{, }\DecValTok{1}\NormalTok{), }\CommentTok{\# 0 = mujer y universidad privada, 1 = hombre y universidad pública}
    \AttributeTok{salario =} \FunctionTok{c}\NormalTok{(}\FunctionTok{coef}\NormalTok{(modelo)[}\DecValTok{1}\NormalTok{], }\FunctionTok{coef}\NormalTok{(modelo)[}\DecValTok{1}\NormalTok{] }\SpecialCharTok{+} \FunctionTok{coef}\NormalTok{(modelo)[}\DecValTok{2}\NormalTok{] }\SpecialCharTok{+} \FunctionTok{coef}\NormalTok{(modelo)[}\DecValTok{3}\NormalTok{])}
\NormalTok{  )}
  \FunctionTok{return}\NormalTok{(recta)}
\NormalTok{\}}

\CommentTok{\# Realizar la simulación de Monte Carlo}
\NormalTok{resultados }\OtherTok{\textless{}{-}} \FunctionTok{lapply}\NormalTok{(}\DecValTok{1}\SpecialCharTok{:}\DecValTok{10}\NormalTok{, realizar\_regresion\_y\_guardar\_recta\_2)}

\CommentTok{\# Graficar todas las rectas de regresión en un solo gráfico}
\FunctionTok{plot}\NormalTok{(}\ConstantTok{NULL}\NormalTok{, }\AttributeTok{xlim =} \FunctionTok{c}\NormalTok{(}\DecValTok{0}\NormalTok{, }\DecValTok{1}\NormalTok{), }\AttributeTok{ylim =} \FunctionTok{c}\NormalTok{(}\DecValTok{290000}\NormalTok{, }\DecValTok{400000}\NormalTok{), }\AttributeTok{xlab =} \StringTok{"Interacción Género{-}Universidad"}\NormalTok{, }\AttributeTok{ylab =} \StringTok{"Salario"}\NormalTok{, }\AttributeTok{main =} \StringTok{"Rectas de Regresión"}\NormalTok{)}
\ControlFlowTok{for}\NormalTok{ (i }\ControlFlowTok{in} \DecValTok{1}\SpecialCharTok{:}\DecValTok{10}\NormalTok{) \{}
  \FunctionTok{lines}\NormalTok{(resultados[[i]]}\SpecialCharTok{$}\NormalTok{interaccion, resultados[[i]]}\SpecialCharTok{$}\NormalTok{salario, }\AttributeTok{col =}\NormalTok{ i)}
\NormalTok{\}}
\end{Highlighting}
\end{Shaded}

\includegraphics{tp2V3_files/figure-latex/unnamed-chunk-8-1.pdf}

\begin{Shaded}
\begin{Highlighting}[]
\CommentTok{\#legend("topright", legend = paste("Regresión", 1:10), col = 1:10, lty = 1)}
\end{Highlighting}
\end{Shaded}

Buscando encontrar otra forma de representar gráficamente la recta de
regresión de salario, en eje x se muestra la iteraccion
Genero-universidad.

\begin{itemize}
\tightlist
\item
  El eje x representa la interacción entre género y universidad (0 =
  mujer y universidad privada, 1 = hombre y universidad pública).
\item
  El eje y representa el salario (de 250 a 400 mil unidades monetarias).
  - Cada recta de regresión se grafica con un color diferente (del 1 al
  10).
\end{itemize}

\end{document}
