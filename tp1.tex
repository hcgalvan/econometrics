% Options for packages loaded elsewhere
\PassOptionsToPackage{unicode}{hyperref}
\PassOptionsToPackage{hyphens}{url}
%
\documentclass[
]{article}
\usepackage{amsmath,amssymb}
\usepackage{iftex}
\ifPDFTeX
  \usepackage[T1]{fontenc}
  \usepackage[utf8]{inputenc}
  \usepackage{textcomp} % provide euro and other symbols
\else % if luatex or xetex
  \usepackage{unicode-math} % this also loads fontspec
  \defaultfontfeatures{Scale=MatchLowercase}
  \defaultfontfeatures[\rmfamily]{Ligatures=TeX,Scale=1}
\fi
\usepackage{lmodern}
\ifPDFTeX\else
  % xetex/luatex font selection
\fi
% Use upquote if available, for straight quotes in verbatim environments
\IfFileExists{upquote.sty}{\usepackage{upquote}}{}
\IfFileExists{microtype.sty}{% use microtype if available
  \usepackage[]{microtype}
  \UseMicrotypeSet[protrusion]{basicmath} % disable protrusion for tt fonts
}{}
\makeatletter
\@ifundefined{KOMAClassName}{% if non-KOMA class
  \IfFileExists{parskip.sty}{%
    \usepackage{parskip}
  }{% else
    \setlength{\parindent}{0pt}
    \setlength{\parskip}{6pt plus 2pt minus 1pt}}
}{% if KOMA class
  \KOMAoptions{parskip=half}}
\makeatother
\usepackage{xcolor}
\usepackage[margin=1in]{geometry}
\usepackage{color}
\usepackage{fancyvrb}
\newcommand{\VerbBar}{|}
\newcommand{\VERB}{\Verb[commandchars=\\\{\}]}
\DefineVerbatimEnvironment{Highlighting}{Verbatim}{commandchars=\\\{\}}
% Add ',fontsize=\small' for more characters per line
\usepackage{framed}
\definecolor{shadecolor}{RGB}{248,248,248}
\newenvironment{Shaded}{\begin{snugshade}}{\end{snugshade}}
\newcommand{\AlertTok}[1]{\textcolor[rgb]{0.94,0.16,0.16}{#1}}
\newcommand{\AnnotationTok}[1]{\textcolor[rgb]{0.56,0.35,0.01}{\textbf{\textit{#1}}}}
\newcommand{\AttributeTok}[1]{\textcolor[rgb]{0.13,0.29,0.53}{#1}}
\newcommand{\BaseNTok}[1]{\textcolor[rgb]{0.00,0.00,0.81}{#1}}
\newcommand{\BuiltInTok}[1]{#1}
\newcommand{\CharTok}[1]{\textcolor[rgb]{0.31,0.60,0.02}{#1}}
\newcommand{\CommentTok}[1]{\textcolor[rgb]{0.56,0.35,0.01}{\textit{#1}}}
\newcommand{\CommentVarTok}[1]{\textcolor[rgb]{0.56,0.35,0.01}{\textbf{\textit{#1}}}}
\newcommand{\ConstantTok}[1]{\textcolor[rgb]{0.56,0.35,0.01}{#1}}
\newcommand{\ControlFlowTok}[1]{\textcolor[rgb]{0.13,0.29,0.53}{\textbf{#1}}}
\newcommand{\DataTypeTok}[1]{\textcolor[rgb]{0.13,0.29,0.53}{#1}}
\newcommand{\DecValTok}[1]{\textcolor[rgb]{0.00,0.00,0.81}{#1}}
\newcommand{\DocumentationTok}[1]{\textcolor[rgb]{0.56,0.35,0.01}{\textbf{\textit{#1}}}}
\newcommand{\ErrorTok}[1]{\textcolor[rgb]{0.64,0.00,0.00}{\textbf{#1}}}
\newcommand{\ExtensionTok}[1]{#1}
\newcommand{\FloatTok}[1]{\textcolor[rgb]{0.00,0.00,0.81}{#1}}
\newcommand{\FunctionTok}[1]{\textcolor[rgb]{0.13,0.29,0.53}{\textbf{#1}}}
\newcommand{\ImportTok}[1]{#1}
\newcommand{\InformationTok}[1]{\textcolor[rgb]{0.56,0.35,0.01}{\textbf{\textit{#1}}}}
\newcommand{\KeywordTok}[1]{\textcolor[rgb]{0.13,0.29,0.53}{\textbf{#1}}}
\newcommand{\NormalTok}[1]{#1}
\newcommand{\OperatorTok}[1]{\textcolor[rgb]{0.81,0.36,0.00}{\textbf{#1}}}
\newcommand{\OtherTok}[1]{\textcolor[rgb]{0.56,0.35,0.01}{#1}}
\newcommand{\PreprocessorTok}[1]{\textcolor[rgb]{0.56,0.35,0.01}{\textit{#1}}}
\newcommand{\RegionMarkerTok}[1]{#1}
\newcommand{\SpecialCharTok}[1]{\textcolor[rgb]{0.81,0.36,0.00}{\textbf{#1}}}
\newcommand{\SpecialStringTok}[1]{\textcolor[rgb]{0.31,0.60,0.02}{#1}}
\newcommand{\StringTok}[1]{\textcolor[rgb]{0.31,0.60,0.02}{#1}}
\newcommand{\VariableTok}[1]{\textcolor[rgb]{0.00,0.00,0.00}{#1}}
\newcommand{\VerbatimStringTok}[1]{\textcolor[rgb]{0.31,0.60,0.02}{#1}}
\newcommand{\WarningTok}[1]{\textcolor[rgb]{0.56,0.35,0.01}{\textbf{\textit{#1}}}}
\usepackage{graphicx}
\makeatletter
\def\maxwidth{\ifdim\Gin@nat@width>\linewidth\linewidth\else\Gin@nat@width\fi}
\def\maxheight{\ifdim\Gin@nat@height>\textheight\textheight\else\Gin@nat@height\fi}
\makeatother
% Scale images if necessary, so that they will not overflow the page
% margins by default, and it is still possible to overwrite the defaults
% using explicit options in \includegraphics[width, height, ...]{}
\setkeys{Gin}{width=\maxwidth,height=\maxheight,keepaspectratio}
% Set default figure placement to htbp
\makeatletter
\def\fps@figure{htbp}
\makeatother
\setlength{\emergencystretch}{3em} % prevent overfull lines
\providecommand{\tightlist}{%
  \setlength{\itemsep}{0pt}\setlength{\parskip}{0pt}}
\setcounter{secnumdepth}{-\maxdimen} % remove section numbering
\ifLuaTeX
  \usepackage{selnolig}  % disable illegal ligatures
\fi
\IfFileExists{bookmark.sty}{\usepackage{bookmark}}{\usepackage{hyperref}}
\IfFileExists{xurl.sty}{\usepackage{xurl}}{} % add URL line breaks if available
\urlstyle{same}
\hypersetup{
  pdftitle={Practical work No 1 - Econometrics Curse, 2024},
  hidelinks,
  pdfcreator={LaTeX via pandoc}}

\title{Practical work No 1 - Econometrics Curse, 2024}
\author{}
\date{\vspace{-2.5em}}

\begin{document}
\maketitle

\hypertarget{equipo}{%
\subsection{Equipo:}\label{equipo}}

\begin{itemize}
\item
  Galván, Hugo César
\item
  García, José Manuel
\end{itemize}

\hypertarget{supponga-que-desea-estimar-el-siguiente-modelo-de-regresiuxf3n-lineal-simple}{%
\paragraph{1. Supponga que desea estimar el siguiente modelo de
regresión lineal
simple}\label{supponga-que-desea-estimar-el-siguiente-modelo-de-regresiuxf3n-lineal-simple}}

Y = β0 + β1X + ε

Genere dos variables aleatorias del modelo (Y, X) teniendo en cuenta que
el tamaño de muestra sea de 200 y para todos los casos. Tips (elija la
media, desvio estandar que desee. Use distribución normal ).

\begin{Shaded}
\begin{Highlighting}[]
\DocumentationTok{\#\# Generación de muestras X e Y}
\FunctionTok{set.seed}\NormalTok{(}\DecValTok{123}\NormalTok{) }
\CommentTok{\# Definir parámetros}
\NormalTok{n }\OtherTok{\textless{}{-}} \DecValTok{200} \CommentTok{\# cant observ}
\NormalTok{μX }\OtherTok{\textless{}{-}} \DecValTok{10} \CommentTok{\# media de x}
\NormalTok{σX }\OtherTok{\textless{}{-}} \DecValTok{2} \CommentTok{\# desv std x}
\NormalTok{μY }\OtherTok{\textless{}{-}} \DecValTok{15} \CommentTok{\# media de y}
\NormalTok{σY }\OtherTok{\textless{}{-}} \DecValTok{3} \CommentTok{\# desv std y}

\CommentTok{\# Generar random variables dependientes}
\NormalTok{Z1 }\OtherTok{\textless{}{-}} \FunctionTok{rnorm}\NormalTok{(n)}
\NormalTok{Z2 }\OtherTok{\textless{}{-}} \FunctionTok{rnorm}\NormalTok{(n)}

\NormalTok{x }\OtherTok{\textless{}{-}}\NormalTok{ μX }\SpecialCharTok{+}\NormalTok{ σX }\SpecialCharTok{*}\NormalTok{ Z1}
\NormalTok{y }\OtherTok{\textless{}{-}}\NormalTok{ μY }\SpecialCharTok{+}\NormalTok{ σY }\SpecialCharTok{*}\NormalTok{ Z2}

\NormalTok{dat }\OtherTok{\textless{}{-}} \FunctionTok{cbind}\NormalTok{(x,y) }\CommentTok{\#junto los valores en una lista}
\end{Highlighting}
\end{Shaded}

\hypertarget{dibuje-un-diagrama-de-dispersiuxf3n-de-las-variabes-simuladas.-quuxe9-observa.-alguna-relaciuxf3n-lineal-se-observa}{%
\paragraph{2. Dibuje un diagrama de dispersión de las variabes
simuladas. ¿Qué observa?.¿ Alguna relación lineal se
observa?}\label{dibuje-un-diagrama-de-dispersiuxf3n-de-las-variabes-simuladas.-quuxe9-observa.-alguna-relaciuxf3n-lineal-se-observa}}

\begin{Shaded}
\begin{Highlighting}[]
\CommentTok{\# library(ggplot2)}
\CommentTok{\# pairs(dat) \# otra forma de verificar relación, una matriz de diagramas de dispersión}
\FunctionTok{plot}\NormalTok{(dat)}
\end{Highlighting}
\end{Shaded}

\includegraphics{tp1_files/figure-latex/unnamed-chunk-2-1.pdf}

Se observa una baja o nula relación entre las variables x e y. No se
percibe un patrón lineal entre ambas variables.

\hypertarget{realice-una-regresiuxf3n-lineal-convencional-y-muestre-sus-resultados.-interprete-la-significatividad-individual-y-grupal-del-modelo.}{%
\paragraph{3. Realice una regresión lineal convencional y muestre sus
resultados. Interprete la significatividad individual y grupal del
modelo.}\label{realice-una-regresiuxf3n-lineal-convencional-y-muestre-sus-resultados.-interprete-la-significatividad-individual-y-grupal-del-modelo.}}

\begin{Shaded}
\begin{Highlighting}[]
\CommentTok{\# Realizo regresión lineal }
\NormalTok{regresion }\OtherTok{\textless{}{-}} \FunctionTok{lm}\NormalTok{(y }\SpecialCharTok{\textasciitilde{}}\NormalTok{ x, }\AttributeTok{data =} \FunctionTok{data.frame}\NormalTok{(dat))}
\FunctionTok{summary}\NormalTok{(regresion)}
\end{Highlighting}
\end{Shaded}

\begin{verbatim}
## 
## Call:
## lm(formula = y ~ x, data = data.frame(dat))
## 
## Residuals:
##     Min      1Q  Median      3Q     Max 
## -7.4376 -1.9686 -0.1022  2.0411  7.5559 
## 
## Coefficients:
##             Estimate Std. Error t value Pr(>|t|)    
## (Intercept) 15.56443    1.14305   13.62   <2e-16 ***
## x           -0.04388    0.11252   -0.39    0.697    
## ---
## Signif. codes:  0 '***' 0.001 '**' 0.01 '*' 0.05 '.' 0.1 ' ' 1
## 
## Residual standard error: 2.994 on 198 degrees of freedom
## Multiple R-squared:  0.0007675,  Adjusted R-squared:  -0.004279 
## F-statistic: 0.1521 on 1 and 198 DF,  p-value: 0.697
\end{verbatim}

\begin{Shaded}
\begin{Highlighting}[]
\CommentTok{\#boxplot(regresion$residuals)}
\end{Highlighting}
\end{Shaded}

En modelos lineales simples, dado que solo hay un predictor,
el~\emph{p-value}~del test F es igual
al~\emph{p-value}~del~\emph{t-test}~del predictor.

Significatividad individual:

\begin{itemize}
\item
  La primera columna (\emph{Estimate}) devuelve el valor estimado para
  los dos parámetros de la ecuación del modelo lineal (b0 y b1) que
  equivalen a la ordenada en el origen y la pendiente.
\item
  Se muestran los errores estándar, el valor del estadístico \emph{t} y
  el \emph{p-value} (dos colas) de cada uno de los dos parámetros. Esto
  permite determinar si los parámetros son significativamente distintos
  de 0, es decir, que tengan importancia en el modelo. En los modelos de
  regresión lineal simple, el parámetro más informativo suele ser la
  pendiente.
\item
  Para el modelo generado, la ordenada en el origen (intercept) es
  significativa, y la pendiente b1 no es significativa (\emph{p-values}
  \textgreater{} 0.1).
\end{itemize}

Significatividad Grupal del model

\begin{itemize}
\item
  El valor de r2 indica que el modelo calculado explica el -0.004\% de
  la variabilidad presente en la variable respuesta (\emph{y}) mediante
  la variable independiente (\emph{x}).
\item
  El \emph{p-value} obtenido en el test F (0.697) determina que no es
  significativamente superior la varianza explicada por el modelo en
  comparación a la varianza total. Es el parámetro que determina si el
  modelo es significativo y por lo tanto no es aceptable.
\end{itemize}

\hypertarget{agregue-la-linea-de-regresiuxf3n-al-diagrama-de-dispersiuxf3n-que-graficuxf3-en-el-punto-1.-quuxe9-observa}{%
\paragraph{4. Agregue la linea de regresión al diagrama de dispersión
que graficó en el punto 1.¿ qué
observa?}\label{agregue-la-linea-de-regresiuxf3n-al-diagrama-de-dispersiuxf3n-que-graficuxf3-en-el-punto-1.-quuxe9-observa}}

\begin{Shaded}
\begin{Highlighting}[]
\CommentTok{\# library(psych)}
\CommentTok{\# library(GGally)}
\CommentTok{\#ggpairs(dat, aes(alpha = 0.5),lower = list(continuous = "smooth"))}

\CommentTok{\# Creamos el gráfico}
\FunctionTok{plot}\NormalTok{(x, y)}
\CommentTok{\# Línea de regresión}
\FunctionTok{abline}\NormalTok{(}\FunctionTok{lm}\NormalTok{(y }\SpecialCharTok{\textasciitilde{}}\NormalTok{ x))}
\end{Highlighting}
\end{Shaded}

\includegraphics{tp1_files/figure-latex/unnamed-chunk-4-1.pdf}

Se observa una linea prácticamente horizontal y una dispersión de puntos
que se alejan de la recta.

\hypertarget{en-caso-de-que-no-observe-relaciuxf3n-entre-las-variables-como-podruxedaa-solucionarlo-tips-puede-resolver-modificando-el-punto-1-y-repitiendo-todos-los-puntos-2-3-y-4}{%
\paragraph{5. En caso de que no observe relación entre las variables
¿Como podríaa solucionarlo? Tips (puede resolver modificando el punto 1
y repitiendo todos los puntos , 2, 3 y
4)}\label{en-caso-de-que-no-observe-relaciuxf3n-entre-las-variables-como-podruxedaa-solucionarlo-tips-puede-resolver-modificando-el-punto-1-y-repitiendo-todos-los-puntos-2-3-y-4}}

\begin{Shaded}
\begin{Highlighting}[]
\FunctionTok{set.seed}\NormalTok{(}\DecValTok{123}\NormalTok{)  }\CommentTok{\# Set a seed for reproducibility}

\CommentTok{\# Definir parámetros}
\NormalTok{n }\OtherTok{\textless{}{-}} \DecValTok{200} \CommentTok{\# cant observ}
\NormalTok{μX }\OtherTok{\textless{}{-}} \DecValTok{10}
\NormalTok{σX }\OtherTok{\textless{}{-}} \DecValTok{2}
\NormalTok{μY }\OtherTok{\textless{}{-}} \DecValTok{15}
\NormalTok{σY }\OtherTok{\textless{}{-}} \DecValTok{3}
\NormalTok{ρ }\OtherTok{\textless{}{-}} \FloatTok{0.7} \CommentTok{\#grado de correlación}

\CommentTok{\# Generar random variables dependientes}
\NormalTok{Z1 }\OtherTok{\textless{}{-}} \FunctionTok{rnorm}\NormalTok{(n)}
\NormalTok{Z2 }\OtherTok{\textless{}{-}} \FunctionTok{rnorm}\NormalTok{(n)}

\NormalTok{x }\OtherTok{\textless{}{-}}\NormalTok{ μX }\SpecialCharTok{+}\NormalTok{ σX }\SpecialCharTok{*}\NormalTok{ Z1}
\NormalTok{y }\OtherTok{\textless{}{-}}\NormalTok{ μY }\SpecialCharTok{+}\NormalTok{ σY }\SpecialCharTok{*}\NormalTok{ (ρ }\SpecialCharTok{*}\NormalTok{ x }\SpecialCharTok{+} \FunctionTok{sqrt}\NormalTok{(}\DecValTok{1} \SpecialCharTok{{-}}\NormalTok{ ρ}\SpecialCharTok{\^{}}\DecValTok{2}\NormalTok{) }\SpecialCharTok{*}\NormalTok{ Z2)}
\NormalTok{dat }\OtherTok{\textless{}{-}} \FunctionTok{cbind}\NormalTok{(x,y) }\CommentTok{\#junto los valores en una lista}
\FunctionTok{plot}\NormalTok{(dat)}
\FunctionTok{abline}\NormalTok{(}\FunctionTok{lm}\NormalTok{(y }\SpecialCharTok{\textasciitilde{}}\NormalTok{ x))}
\end{Highlighting}
\end{Shaded}

\includegraphics{tp1_files/figure-latex/unnamed-chunk-5-1.pdf}

\begin{Shaded}
\begin{Highlighting}[]
\CommentTok{\# ggpairs(dat, aes(alpha = 0.5),lower = list(continuous = "smooth"))}
\end{Highlighting}
\end{Shaded}

\begin{Shaded}
\begin{Highlighting}[]
\CommentTok{\# Realizo regresión lineal }
\NormalTok{modelo }\OtherTok{\textless{}{-}} \FunctionTok{lm}\NormalTok{(y }\SpecialCharTok{\textasciitilde{}}\NormalTok{ x, }\AttributeTok{data =} \FunctionTok{data.frame}\NormalTok{(dat))}
\FunctionTok{summary}\NormalTok{(modelo)}
\end{Highlighting}
\end{Shaded}

\begin{verbatim}
## 
## Call:
## lm(formula = y ~ x, data = data.frame(dat))
## 
## Residuals:
##    Min     1Q Median     3Q    Max 
## -5.311 -1.406 -0.073  1.458  5.396 
## 
## Coefficients:
##             Estimate Std. Error t value Pr(>|t|)    
## (Intercept) 15.40308    0.81630   18.87   <2e-16 ***
## x            2.06866    0.08036   25.74   <2e-16 ***
## ---
## Signif. codes:  0 '***' 0.001 '**' 0.01 '*' 0.05 '.' 0.1 ' ' 1
## 
## Residual standard error: 2.138 on 198 degrees of freedom
## Multiple R-squared:   0.77,  Adjusted R-squared:  0.7688 
## F-statistic: 662.7 on 1 and 198 DF,  p-value: < 2.2e-16
\end{verbatim}

\hypertarget{en-base-al-modelo-de-regresiuxf3n-realice-una-predicciuxf3n-para-20-datos-muxe1s.}{%
\paragraph{6. En base al modelo de regresión realice una predicción para
20 datos
más.}\label{en-base-al-modelo-de-regresiuxf3n-realice-una-predicciuxf3n-para-20-datos-muxe1s.}}

\begin{Shaded}
\begin{Highlighting}[]
\FunctionTok{set.seed}\NormalTok{(}\DecValTok{123}\NormalTok{)}
\NormalTok{Z1 }\OtherTok{\textless{}{-}} \FunctionTok{rnorm}\NormalTok{(}\DecValTok{20}\NormalTok{)}
\NormalTok{x }\OtherTok{\textless{}{-}}\NormalTok{ μX }\SpecialCharTok{+}\NormalTok{ σX }\SpecialCharTok{*}\NormalTok{ Z1}
\FunctionTok{predict}\NormalTok{(}\AttributeTok{object=}\NormalTok{modelo, }\AttributeTok{newdata=}\FunctionTok{data.frame}\NormalTok{(x))}
\end{Highlighting}
\end{Shaded}

\begin{verbatim}
##        1        2        3        4        5        6        7        8 
## 33.77083 35.13739 42.53859 36.38142 36.62461 43.18548 37.99666 30.85574 
##        9       10       11       12       13       14       15       16 
## 33.24797 34.24586 41.15413 37.57837 37.74783 36.54763 33.79001 43.48274 
##       17       18       19       20 
## 38.14947 27.95317 38.99144 34.13361
\end{verbatim}

\hypertarget{simule-mediante-montecarlo-1000-regresiones-lineales-tomando-de-base-el-modelo-de-la-ecuaciuxf3n-1.-tips-en-un-primer-paso-coloque-valores-numuxe9ricos-de-los-coeficientes-a-estimar-intercepto-y-beta-ademuxe1s-un-valor-para-el-desvuxedo-estuxe1ndar-del-error}{%
\paragraph{7. Simule mediante Montecarlo 1000 regresiones lineales
tomando de base el modelo de la ecuación (1). TIPS: en un primer paso
coloque valores numéricos de los coeficientes a estimar (intercepto , y
beta) además un valor para el desvío estándar del
error}\label{simule-mediante-montecarlo-1000-regresiones-lineales-tomando-de-base-el-modelo-de-la-ecuaciuxf3n-1.-tips-en-un-primer-paso-coloque-valores-numuxe9ricos-de-los-coeficientes-a-estimar-intercepto-y-beta-ademuxe1s-un-valor-para-el-desvuxedo-estuxe1ndar-del-error}}

\begin{Shaded}
\begin{Highlighting}[]
\FunctionTok{set.seed}\NormalTok{(}\DecValTok{123}\NormalTok{)}
\CommentTok{\# Establecer numero de simulaciones}
\NormalTok{num\_simulations }\OtherTok{\textless{}{-}} \DecValTok{1000}
\NormalTok{n }\OtherTok{\textless{}{-}} \DecValTok{100}
\CommentTok{\# Establecer parametros del modelo}
\NormalTok{intercept }\OtherTok{\textless{}{-}} \FloatTok{4.9}
\NormalTok{beta }\OtherTok{\textless{}{-}} \FloatTok{1.02}
\NormalTok{error\_std\_dev }\OtherTok{\textless{}{-}} \FloatTok{2.138}

\CommentTok{\# Inicializar vectores que contendrán los coeficientes estimados}
\NormalTok{pendientes }\OtherTok{\textless{}{-}} \FunctionTok{vector}\NormalTok{(}\AttributeTok{length =}\NormalTok{ num\_simulations)}
\NormalTok{intercepts }\OtherTok{\textless{}{-}} \FunctionTok{vector}\NormalTok{(}\AttributeTok{length =}\NormalTok{ num\_simulations)}

\CommentTok{\# Perform the simulations}
\ControlFlowTok{for}\NormalTok{ (i }\ControlFlowTok{in} \DecValTok{1}\SpecialCharTok{:}\NormalTok{num\_simulations) \{}
  \CommentTok{\# Generate the independent variable (x)}
\NormalTok{  Z1 }\OtherTok{\textless{}{-}} \FunctionTok{rnorm}\NormalTok{(n)}
\NormalTok{  x }\OtherTok{\textless{}{-}}\NormalTok{ μX }\SpecialCharTok{+}\NormalTok{ σX }\SpecialCharTok{*}\NormalTok{ Z1}
  
  \CommentTok{\# Generate the dependent variable (y) using the model equation}
\NormalTok{  y }\OtherTok{\textless{}{-}}\NormalTok{ intercept }\SpecialCharTok{+}\NormalTok{ beta }\SpecialCharTok{*}\NormalTok{ x }\SpecialCharTok{+} \FunctionTok{rnorm}\NormalTok{(n, }\DecValTok{0}\NormalTok{, error\_std\_dev)}
  \CommentTok{\#μY + σY * (ρ * Z1 + sqrt(1 {-} ρ\^{}2) * Z2)}
  
  \CommentTok{\# Fit the linear regression model}
\NormalTok{  model }\OtherTok{\textless{}{-}} \FunctionTok{lm}\NormalTok{(y }\SpecialCharTok{\textasciitilde{}}\NormalTok{ x)}
\NormalTok{  pendientes[i] }\OtherTok{\textless{}{-}}\NormalTok{ model}\SpecialCharTok{$}\NormalTok{coefficients[}\DecValTok{2}\NormalTok{]}
\NormalTok{  intercepts[i] }\OtherTok{\textless{}{-}}\NormalTok{ model}\SpecialCharTok{$}\NormalTok{coefficients[}\DecValTok{1}\NormalTok{]}
\NormalTok{\}}
\end{Highlighting}
\end{Shaded}

\hypertarget{realice-un-histograma-para-las-estimaciones-de-la-pendiente-y-del-intercepto.-indique-lo-que-observa.}{%
\paragraph{8. Realice un histograma para las estimaciones de la
pendiente y del intercepto. Indique lo que
observa.}\label{realice-un-histograma-para-las-estimaciones-de-la-pendiente-y-del-intercepto.-indique-lo-que-observa.}}

\begin{Shaded}
\begin{Highlighting}[]
\FunctionTok{par}\NormalTok{(}\AttributeTok{mfrow =} \FunctionTok{c}\NormalTok{(}\DecValTok{1}\NormalTok{, }\DecValTok{2}\NormalTok{))}
\FunctionTok{hist}\NormalTok{(pendientes, }\AttributeTok{main =} \StringTok{"Distribución de Pendiente"}\NormalTok{,}\AttributeTok{xlab =} \StringTok{"Pendiente"}\NormalTok{)}
\FunctionTok{abline}\NormalTok{(}\AttributeTok{v =} \FunctionTok{mean}\NormalTok{(pendientes), }\AttributeTok{col=}\StringTok{"red"}\NormalTok{, }\AttributeTok{lwd=}\DecValTok{3}\NormalTok{, }\AttributeTok{lty=}\DecValTok{2}\NormalTok{)}
\FunctionTok{hist}\NormalTok{(intercepts, }\AttributeTok{main =} \StringTok{"Distribución de Intercept"}\NormalTok{, }\AttributeTok{xlab =} \StringTok{"Intercept"}\NormalTok{)}
\FunctionTok{abline}\NormalTok{(}\AttributeTok{v =} \FunctionTok{mean}\NormalTok{(intercepts), }\AttributeTok{col=}\StringTok{"red"}\NormalTok{, }\AttributeTok{lwd=}\DecValTok{3}\NormalTok{, }\AttributeTok{lty=}\DecValTok{2}\NormalTok{)}
\end{Highlighting}
\end{Shaded}

\includegraphics{tp1_files/figure-latex/unnamed-chunk-10-1.pdf}

\hypertarget{realice-un-gruxe1fico-de-densidad-para-las-estimaciones-de-la-pendiente-y-del-intercepto.-indique-lo-que-observa.}{%
\paragraph{9. Realice un gráfico de densidad para las estimaciones de la
pendiente y del intercepto. Indique lo que
observa.}\label{realice-un-gruxe1fico-de-densidad-para-las-estimaciones-de-la-pendiente-y-del-intercepto.-indique-lo-que-observa.}}

\begin{Shaded}
\begin{Highlighting}[]
\FunctionTok{par}\NormalTok{(}\AttributeTok{mfrow =} \FunctionTok{c}\NormalTok{(}\DecValTok{1}\NormalTok{, }\DecValTok{2}\NormalTok{))}
\FunctionTok{plot}\NormalTok{(}\FunctionTok{density}\NormalTok{(pendientes, }\AttributeTok{width =}\NormalTok{ .}\DecValTok{2}\NormalTok{))}
\FunctionTok{plot}\NormalTok{(}\FunctionTok{density}\NormalTok{(intercepts, }\AttributeTok{width =}\NormalTok{ .}\DecValTok{2}\NormalTok{))}
\end{Highlighting}
\end{Shaded}

\includegraphics{tp1_files/figure-latex/unnamed-chunk-11-1.pdf}

\hypertarget{repita-la-simulaciuxf3n-de-montecarlo-pero-sin-que-la-variable-x-tenga-una-distribuciuxf3n-normal-por-ejemplo-una-distribuciuxf3n-uniforme.-considere-los-puntos-7-8-9-de-este-tp.}{%
\paragraph{10. Repita la simulación de montecarlo pero sin que la
variable X tenga una distribución normal, por ejemplo una distribución
uniforme. Considere los puntos 7, 8, 9 de este
TP.}\label{repita-la-simulaciuxf3n-de-montecarlo-pero-sin-que-la-variable-x-tenga-una-distribuciuxf3n-normal-por-ejemplo-una-distribuciuxf3n-uniforme.-considere-los-puntos-7-8-9-de-este-tp.}}

\begin{Shaded}
\begin{Highlighting}[]
\CommentTok{\# Establecer numero de simulaciones}
\FunctionTok{set.seed}\NormalTok{(}\DecValTok{123}\NormalTok{)}
\NormalTok{num\_simulations }\OtherTok{\textless{}{-}} \DecValTok{1000}
\NormalTok{n }\OtherTok{\textless{}{-}} \DecValTok{100}
\CommentTok{\# Establecer parametros del modelo}
\NormalTok{intercept }\OtherTok{\textless{}{-}} \FloatTok{4.9}
\NormalTok{beta }\OtherTok{\textless{}{-}} \DecValTok{2}
\NormalTok{error\_std\_dev }\OtherTok{\textless{}{-}} \DecValTok{1}

\CommentTok{\# Inicializar vectores que contendrán los coeficientes estimados}
\NormalTok{pendientes }\OtherTok{\textless{}{-}} \FunctionTok{vector}\NormalTok{(}\AttributeTok{length =}\NormalTok{ num\_simulations)}
\NormalTok{intercepts }\OtherTok{\textless{}{-}} \FunctionTok{vector}\NormalTok{(}\AttributeTok{length =}\NormalTok{ num\_simulations)}

\CommentTok{\# Perform the simulations}
\ControlFlowTok{for}\NormalTok{ (i }\ControlFlowTok{in} \DecValTok{1}\SpecialCharTok{:}\NormalTok{num\_simulations) \{}
  \CommentTok{\# Generate the independent variable (x)}
\NormalTok{  x }\OtherTok{\textless{}{-}} \FunctionTok{runif}\NormalTok{(n)}
  \CommentTok{\# plot(x)}
  
  \CommentTok{\# Generate the dependent variable (y) using the model equation}
\NormalTok{  y }\OtherTok{\textless{}{-}}\NormalTok{ intercept }\SpecialCharTok{+}\NormalTok{ beta }\SpecialCharTok{*}\NormalTok{ x }\SpecialCharTok{+} \FunctionTok{rnorm}\NormalTok{(n, }\DecValTok{0}\NormalTok{, error\_std\_dev)}
  
  \CommentTok{\# Fit the linear regression model}
\NormalTok{  model }\OtherTok{\textless{}{-}} \FunctionTok{lm}\NormalTok{(y }\SpecialCharTok{\textasciitilde{}}\NormalTok{ x)}
\NormalTok{  pendientes[i] }\OtherTok{\textless{}{-}}\NormalTok{ model}\SpecialCharTok{$}\NormalTok{coefficients[}\DecValTok{2}\NormalTok{]}
\NormalTok{  intercepts[i] }\OtherTok{\textless{}{-}}\NormalTok{ model}\SpecialCharTok{$}\NormalTok{coefficients[}\DecValTok{1}\NormalTok{]}
\NormalTok{\}}

\FunctionTok{par}\NormalTok{(}\AttributeTok{mfrow =} \FunctionTok{c}\NormalTok{(}\DecValTok{2}\NormalTok{, }\DecValTok{2}\NormalTok{))}
\FunctionTok{hist}\NormalTok{(pendientes, }\AttributeTok{main =} \StringTok{"Distribución de Pendiente Estimadas"}\NormalTok{,}\AttributeTok{xlab =} \StringTok{"Pendiente"}\NormalTok{)}
\FunctionTok{abline}\NormalTok{(}\AttributeTok{v =} \FunctionTok{mean}\NormalTok{(pendientes), }\AttributeTok{col=}\StringTok{"red"}\NormalTok{, }\AttributeTok{lwd=}\DecValTok{3}\NormalTok{, }\AttributeTok{lty=}\DecValTok{2}\NormalTok{)}
\FunctionTok{hist}\NormalTok{(intercepts, }\AttributeTok{main =} \StringTok{"Distribución de Intercept Estimados"}\NormalTok{, }\AttributeTok{xlab =} \StringTok{"Intercept"}\NormalTok{)}
\FunctionTok{abline}\NormalTok{(}\AttributeTok{v =} \FunctionTok{mean}\NormalTok{(intercepts), }\AttributeTok{col=}\StringTok{"red"}\NormalTok{, }\AttributeTok{lwd=}\DecValTok{3}\NormalTok{, }\AttributeTok{lty=}\DecValTok{2}\NormalTok{)}
\FunctionTok{plot}\NormalTok{(}\FunctionTok{density}\NormalTok{(pendientes, }\AttributeTok{width =}\NormalTok{ .}\DecValTok{2}\NormalTok{))}
\FunctionTok{plot}\NormalTok{(}\FunctionTok{density}\NormalTok{(intercepts, }\AttributeTok{width =}\NormalTok{ .}\DecValTok{2}\NormalTok{))}
\end{Highlighting}
\end{Shaded}

\includegraphics{tp1_files/figure-latex/unnamed-chunk-12-1.pdf}

\end{document}
